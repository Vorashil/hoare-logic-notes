% Section 11: Rules of Consequence
\section{Rules of Consequence}

\begin{frame}{Precondition Strengthening}
    \begin{block}{Recall}
        Recall that
        \[ \frac{\vdash S_1, \ldots, \vdash S_n}{\vdash S} \]
        means $\vdash S$ can be deduced from $\vdash S_1, \ldots, \vdash S_n$
    \end{block}

\end{frame}

\begin{frame}{Precondition Strengthening}
    \begin{block}{The Rule}
        Using this notation, the rule of \textbf{precondition strengthening} is:
        \[ \frac{\vdash P \Rightarrow P', \quad \vdash \{P'\} \; C \; \{Q\}}{\vdash \{P\} \; C \; \{Q\}} \]
    \end{block}
    \begin{alertblock}{Note}
        The two hypotheses are different kinds of judgements:
        \begin{itemize}
            \item $\vdash P \Rightarrow P'$ is a mathematical/logical judgement
            \item $\vdash \{P'\} \; C \; \{Q\}$ is a program correctness judgement
        \end{itemize}
    \end{alertblock}
\end{frame}

\begin{frame}{Understanding Precondition Strengthening}
    \begin{block}{What Does This Rule Mean?}
        \begin{itemize}
            \item If $P$ implies $P'$ (i.e., $P$ is stronger than $P'$)
            \item And we know that $\{P'\} \; C \; \{Q\}$ holds
            \item Then $\{P\} \; C \; \{Q\}$ also holds
        \end{itemize}
    \end{block}

    \begin{block}{Intuition}
        \begin{itemize}
            \item A stronger precondition gives us more information
            \item If the program works correctly with less information ($P'$)
            \item It will certainly work with more information ($P$)
            \item ``Demanding more from the input never hurts''
        \end{itemize}
    \end{block}


\end{frame}
\begin{frame}{Understanding Precondition Strengthening}
    \begin{example}[Simple Example]
        \begin{itemize}
            \item Know: $\vdash \{x > 0\} \; y := x \; \{y > 0\}$
            \item Have: $x = 5 \Rightarrow x > 0$
            \item Conclude: $\vdash \{x = 5\} \; y := x \; \{y > 0\}$
        \end{itemize}
    \end{example}
\end{frame}

\begin{frame}{Postcondition Weakening}
    \begin{block}{The Dual Rule}
        Just as the previous rule allows the precondition of a partial correctness specification to be strengthened, the following one allows us to weaken the postcondition
    \end{block}

    \begin{block}{Postcondition Weakening Rule}
        \[ \frac{\vdash \{P\} \; C \; \{Q'\}, \quad \vdash Q' \Rightarrow Q}{\vdash \{P\} \; C \; \{Q\}} \]
    \end{block}
\end{frame}

\begin{frame}{Understanding Postcondition Weakening}
    \begin{block}{What Does This Rule Mean?}
        \begin{itemize}
            \item If we can establish $\{P\} \; C \; \{Q'\}$
            \item And $Q'$ implies $Q$ (i.e., $Q'$ is stronger than $Q$)
            \item Then $\{P\} \; C \; \{Q\}$ also holds
        \end{itemize}
    \end{block}

    \begin{block}{Intuition}
        \begin{itemize}
            \item If the program establishes a strong property ($Q'$)
            \item It automatically establishes any weaker property ($Q$)
            \item ``Promising less in the output is always safe''
        \end{itemize}
    \end{block}

    \begin{example}[Simple Example]
        \begin{itemize}
            \item Know: $\vdash \{x = 5\} \; y := x + 1 \; \{y = 6\}$
            \item Have: $y = 6 \Rightarrow y > 0$
            \item Conclude: $\vdash \{x = 5\} \; y := x + 1 \; \{y > 0\}$
        \end{itemize}
    \end{example}
\end{frame}

\begin{frame}{The Rule of Consequence}
    \begin{block}{Combining Both Rules}
        Often we use both precondition strengthening and postcondition weakening together. This gives us the general \textbf{rule of consequence}:

        \[ \frac{\vdash P \Rightarrow P', \quad \vdash \{P'\} \; C \; \{Q'\}, \quad \vdash Q' \Rightarrow Q}{\vdash \{P\} \; C \; \{Q\}} \]
    \end{block}

    \begin{block}{When to Use}
        This rule is essential for:
        \begin{itemize}
            \item Adapting existing proofs to new situations
            \item Simplifying complex preconditions or postconditions
            \item Connecting different parts of a larger proof
        \end{itemize}
    \end{block}
\end{frame}

\begin{frame}{Example: Using the Rule of Consequence}
    \begin{example}[Complete Example]
        Want to prove: $\vdash \{x = 10 \wedge y = 5\} \; z := x - y \; \{z > 0\}$

        \begin{enumerate}
            \item By assignment axiom:
            \[\vdash \{x - y > 0\} \; z := x - y \; \{z > 0\}\]

            \item We need to show: $x = 10 \wedge y = 5 \Rightarrow x - y > 0$
            \begin{itemize}
                \item If $x = 10$ and $y = 5$, then $x - y = 5$
                \item And $5 > 0$ is true
            \end{itemize}

            \item By precondition strengthening:
            \[\vdash \{x = 10 \wedge y = 5\} \; z := x - y \; \{z > 0\}\]
        \end{enumerate}
    \end{example}
\end{frame}

\begin{frame}{Why These Rules Matter}
    \begin{block}{Practical Importance}
        The rules of consequence are crucial because:
        \begin{itemize}
            \item Real programs rarely have specifications that match axioms exactly
            \item We need to adapt and combine different proof rules
            \item They allow modular reasoning about programs
        \end{itemize}
    \end{block}

    \begin{block}{Key Insight}
        These rules formalize the intuition that:
        \begin{itemize}
            \item \textbf{Preconditions}: ``If it works with less, it works with more''
            \item \textbf{Postconditions}: ``If it achieves more, it achieves less''
        \end{itemize}
    \end{block}
\end{frame}
\begin{frame}{Why These Rules Matter}

    \begin{alertblock}{Remember}
        The direction matters:
        \begin{itemize}
            \item Preconditions can be \emph{strengthened} (made more specific)
            \item Postconditions can be \emph{weakened} (made more general)
        \end{itemize}
    \end{alertblock}
\end{frame}