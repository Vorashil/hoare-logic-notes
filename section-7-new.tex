% Section 7: Floyd-Hoare Logic
\section{Floyd-Hoare Logic}

\begin{frame}{Floyd-Hoare Logic}
    \begin{block}{The Need for Formal Proofs}
        To construct formal proofs of partial correctness specifications, \emph{axioms} and \emph{rules of inference} are needed
    \end{block}
    
    \begin{block}{What Floyd-Hoare Logic Provides}
        This is what Floyd-Hoare logic provides:
        \begin{itemize}
            \item The formulation of the deductive system is due to Hoare
            \item Some of the underlying ideas originated with Floyd
        \end{itemize}
    \end{block}
\end{frame}

\begin{frame}{Structure of Proofs in Floyd-Hoare Logic}
    \begin{block}{Proof Definition}
        A proof in Floyd-Hoare logic is a sequence of lines, each of which is either:
        \begin{itemize}
            \item An \textbf{axiom} of the logic, or
            \item Follows from earlier lines by a \textbf{rule of inference} of the logic
        \end{itemize}
        
        Note: Proofs can also be trees, if you prefer
    \end{block}
    
    \begin{block}{Purpose of Formal Proofs}
        A formal proof makes explicit what axioms and rules of inference are used to arrive at a conclusion
    \end{block}
\end{frame}

\begin{frame}{Components of Floyd-Hoare Logic}
    \begin{block}{Axioms}
        \textbf{Axioms} are basic facts about specific programming constructs that require no proof:
        \begin{itemize}
            \item Assignment axiom
            \item Skip axiom (for the empty command)
            \item Other basic command axioms
        \end{itemize}
    \end{block}
    
    \begin{block}{Rules of Inference}
        \textbf{Rules of inference} allow us to derive new facts from existing ones:
        \begin{itemize}
            \item Sequence rule (composition)
            \item Conditional rule (if-then-else)
            \item While loop rule
            \item Consequence rule
        \end{itemize}
    \end{block}
\end{frame}

\begin{frame}{Historical Context}
    \begin{block}{Robert W. Floyd (1936-2001)}
        \begin{itemize}
            \item Introduced flowchart-based verification methods (1967)
            \item Pioneered the use of loop invariants
            \item Developed techniques for proving program termination
        \end{itemize}
    \end{block}
    
    \begin{block}{C.A.R. Hoare (1934-)}
        \begin{itemize}
            \item Formalized Floyd's ideas into a logical system (1969)
            \item Introduced the triple notation $\{P\} C \{Q\}$
            \item Created the axiomatic semantics approach
        \end{itemize}
    \end{block}
    
    \begin{alertblock}{Note}
        The system is called ``Floyd-Hoare Logic'' to honor both contributors
    \end{alertblock}
\end{frame}

\begin{frame}{Example: What a Proof Looks Like}
    \begin{example}[Simple Proof Structure]
        To prove $\hoare{x = 5}{y := x + 1; z := y}{z = 6}$:
        
        \begin{enumerate}
            \item $\hoare{x = 5}{y := x + 1}{y = 6}$ \hfill (Assignment axiom)
            \item $\hoare{y = 6}{z := y}{z = 6}$ \hfill (Assignment axiom)
            \item $\hoare{x = 5}{y := x + 1; z := y}{z = 6}$ \hfill (Sequence rule on 1,2)
        \end{enumerate}
        
        Each line is justified by an axiom or rule!
    \end{example}
    
    \begin{block}{Key Insight}
        Floyd-Hoare Logic provides a \emph{systematic} way to prove program correctness, not just intuitive arguments
    \end{block}
\end{frame}