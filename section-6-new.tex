% Section 6: Auxiliary Variables
\section{Auxiliary Variables}

\begin{frame}{Auxiliary Variables}
    \begin{example}[Variable Swap]
        $\hoare{X=x \wedge Y=y}{R:=X;\ X:=Y;\ Y:=R}{X=y \wedge Y=x}$
        
        This says that \emph{if} the execution of
        \begin{center}
            \texttt{R:=X; X:=Y; Y:=R}
        \end{center}
        terminates (which it does)
        
        \emph{then} the values of X and Y are exchanged
    \end{example}
    
    \begin{block}{Key Observation}
        The variables $x$ and $y$, which don't occur in the command and are used to name the initial values of program variables $X$ and $Y$
    \end{block}
\end{frame}

\begin{frame}{What are Auxiliary Variables?}
    \begin{block}{Definition}
        Variables that appear in specifications but not in the program code are called:
        \begin{itemize}
            \item \textbf{Auxiliary variables}
            \item \textbf{Ghost variables}
            \item \textbf{Specification variables}
        \end{itemize}
    \end{block}
    
    \begin{block}{Purpose}
        Auxiliary variables allow us to:
        \begin{itemize}
            \item Refer to initial values of program variables in postconditions
            \item Express relationships between initial and final states
            \item Write more expressive specifications
        \end{itemize}
    \end{block}
\end{frame}

\begin{frame}{Naming Convention}
    \begin{block}{Informal Convention}
        To distinguish between program variables and auxiliary variables:
        \begin{itemize}
            \item \textbf{Program variables} are UPPER CASE (e.g., X, Y, Z)
            \item \textbf{Auxiliary variables} are lower case (e.g., x, y, z)
        \end{itemize}
    \end{block}
    
    \begin{example}[More Examples]
        \begin{itemize}
            \item $\hoare{X=x}{X:=X+1}{X=x+1}$ -- $x$ remembers initial value
            \item $\hoare{X=x \wedge Y=y}{X:=X+Y}{X=x+y \wedge Y=y}$
            \item $\hoare{A[i]=a}{A[i]:=0}{A[i]=0 \wedge \text{``old } A[i] = a\text{''}}$
        \end{itemize}
    \end{example}
\end{frame}

\begin{frame}{Why Auxiliary Variables Matter}
    \begin{block}{Without Auxiliary Variables}
        Consider trying to specify variable swap without auxiliary variables:
        \begin{itemize}
            \item $\hoare{?}{R:=X;\ X:=Y;\ Y:=R}{?}$
            \item How do we say ``X gets Y's initial value''?
            \item We can't refer to initial values!
        \end{itemize}
    \end{block}
    
    \begin{block}{With Auxiliary Variables}
        We can express complex relationships:
        \begin{itemize}
            \item Maximum: $\hoare{X=x \wedge Y=y}{...}{M = \max(x,y)}$
            \item Sorting: $\hoare{A=a}{...}{\text{``A is a sorted permutation of a''}}$
            \item Any computation relating initial and final states
        \end{itemize}
    \end{block}
\end{frame}

\begin{frame}{Important Notes about Auxiliary Variables}
    \begin{alertblock}{Key Properties}
        \begin{enumerate}
            \item Auxiliary variables are \textbf{immutable} -- they never change value during program execution
            \item They exist only in specifications, not in the actual program
            \item They are universally quantified (implicitly)
        \end{enumerate}
    \end{alertblock}
    
    \begin{block}{Formal Interpretation}
        The specification $\hoare{X=x}{C}{Q(x)}$ actually means:
        \[\forall x.\ \hoare{X=x}{C}{Q(x)}\]
        ``For all values $x$, if $X$ starts with value $x$, then after $C$, property $Q(x)$ holds''
    \end{block}
\end{frame}