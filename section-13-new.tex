% Section 13: WHILE Loops and Invariants
\section{WHILE Loops and Invariants}

\begin{frame}{WHILE-commands}
    \begin{block}{Syntax and Semantics}
        \begin{itemize}
            \item \textbf{Syntax}: \texttt{WHILE } $S$ \texttt{ DO } $C$
            \item \textbf{Semantics}:
            \begin{itemize}
                \item If the statement $S$ is true in the current state, then $C$ is executed and the WHILE-command is repeated
                \item If $S$ is false, then nothing is done
                \item Thus $C$ is repeatedly executed until the value of $S$ becomes false
                \item If $S$ never becomes false, then the execution of the command never terminates
            \end{itemize}
        \end{itemize}
    \end{block}
    
    \begin{example}[Simple WHILE Loop]
        \texttt{WHILE $\neg(X=0)$ DO X:= X-2}
        \begin{itemize}
            \item If the value of X is non-zero, then its value is decreased by 2 and then the process is repeated
            \item This WHILE-command will terminate (with X having value 0) if the value of X is an even non-negative number
            \item In all other states it will not terminate
        \end{itemize}
    \end{example}
\end{frame}

\begin{frame}{The Challenge of WHILE Loops}
    \begin{block}{Why WHILE Loops are Difficult}
        \begin{itemize}
            \item Unlike sequence and conditionals, we don't know how many times the loop will execute
            \item We need to reason about \emph{all possible} number of iterations
            \item The loop might not terminate at all!
            \item We need a way to capture what stays true throughout the loop
        \end{itemize}
    \end{block}
    
    \begin{block}{The Key Insight: Invariants}
        \begin{itemize}
            \item An \textbf{invariant} is a property that remains true before and after each iteration
            \item If we can find an appropriate invariant, we can reason about the loop
            \item The invariant captures the ``essence'' of what the loop does
        \end{itemize}
    \end{block}
\end{frame}

\begin{frame}{Invariants}
    \begin{block}{Definition}
        Suppose $\vdash \{P \wedge S\} \; C \; \{P\}$
        
        $P$ is said to be an \emph{invariant of $C$ whenever $S$ holds}
    \end{block}
    
    \begin{block}{The WHILE-rule Intuition}
        The WHILE-rule says that:
        \begin{itemize}
            \item \emph{if} $P$ is an invariant of the body of a WHILE-command whenever the test condition holds
            \item \emph{then} $P$ is an invariant of the whole WHILE-command
        \end{itemize}
    \end{block}
    
    \begin{block}{In Other Words}
        \begin{itemize}
            \item If executing $C$ \emph{once} preserves the truth of $P$
            \item Then executing $C$ \emph{any number of times} also preserves the truth of $P$
        \end{itemize}
    \end{block}
\end{frame}

\begin{frame}{After Termination}
    \begin{block}{What Happens When the Loop Exits?}
        The WHILE-rule also expresses the fact that after a WHILE-command has terminated, the test must be false
        \begin{itemize}
            \item Otherwise, it wouldn't have terminated
            \item So we know both:
            \begin{itemize}
                \item The invariant $P$ still holds
                \item The test condition $S$ is false (i.e., $\neg S$ is true)
            \end{itemize}
        \end{itemize}
    \end{block}
    
    \begin{block}{The Power of Invariants}
        This gives us a powerful way to reason about loops:
        \begin{enumerate}
            \item Find an invariant $P$ that captures the essential property
            \item Prove that $P$ is preserved by the loop body when $S$ is true
            \item Conclude that after the loop, we have $P \wedge \neg S$
        \end{enumerate}
    \end{block}
\end{frame}

\begin{frame}{The WHILE-Rule}
    \begin{block}{The WHILE-rule}
        \[ \frac{\vdash \{P \wedge S\} \; C \; \{P\}}{\vdash \{P\} \; \texttt{WHILE } S \texttt{ DO } C \; \{P \wedge \neg S\}} \]
    \end{block}
    
    \begin{block}{Understanding the Rule}
        \begin{itemize}
            \item \textbf{Premise}: If $P$ and $S$ are both true, then after executing $C$, $P$ is still true
            \item \textbf{Conclusion}: Starting with $P$ true, after the WHILE loop, $P$ is still true AND $S$ is false
            \item The invariant $P$ is maintained throughout all iterations
            \item When the loop exits, we additionally know that $S$ is false
        \end{itemize}
    \end{block}
\end{frame}

\begin{frame}{Example: Integer Division}
    \begin{block}{Goal}
        Prove that the following computes integer division:
    \end{block}
    
    \begin{example}[Division by Repeated Subtraction]
        It is easy to show:
        \[\vdash \{X=R+(Y \times Q) \wedge Y \leq R\} \; R:=R-Y; \; Q:=Q+1 \; \{X=R+(Y \times Q)\}\]
        
        Hence by the WHILE-rule with $P = \text{`}X=R+(Y \times Q)\text{'}$ and $S = \text{`}Y \leq R\text{'}$:
        
        \begin{align*}
        \vdash \{X=R+(Y \times Q)\} \\
        \texttt{WHILE } Y \leq R \texttt{ DO} \\
        \quad (R:=R-Y; \; Q:=Q+1) \\
        \{X=R+(Y \times Q) \wedge \neg(Y \leq R)\}
        \end{align*}
    \end{example}
\end{frame}

\begin{frame}{Analyzing the Division Example}
    \begin{block}{The Invariant}
        $P: X = R + (Y \times Q)$ captures the relationship between:
        \begin{itemize}
            \item $X$: the original dividend
            \item $R$: the current remainder
            \item $Y$: the divisor
            \item $Q$: the quotient being computed
        \end{itemize}
    \end{block}
    
    \begin{block}{Why This Works}
        \begin{itemize}
            \item Initially: $R = X$ and $Q = 0$, so $X = R + (Y \times 0) = R$
            \item Each iteration: We subtract $Y$ from $R$ and add 1 to $Q$
            \item The invariant $X = R + (Y \times Q)$ is preserved
            \item When done: $\neg(Y \leq R)$ means $R < Y$
            \item So we have: $X = R + (Y \times Q)$ with $0 \leq R < Y$
            \item This is exactly the definition of integer division!
        \end{itemize}
    \end{block}
\end{frame}

\begin{frame}{Finding Good Invariants}
    \begin{block}{The Art of Finding Invariants}
        Finding the right invariant is often the hardest part:
        \begin{itemize}
            \item It must be true initially (before the loop starts)
            \item It must be preserved by each iteration
            \item Combined with the negated test, it must imply the desired postcondition
        \end{itemize}
    \end{block}
    
    \begin{block}{Common Patterns}
        \begin{itemize}
            \item \textbf{Accumulation}: Invariant tracks partial results (like sum so far)
            \item \textbf{Bounds}: Invariant maintains bounds on variables
            \item \textbf{Relationships}: Invariant preserves relationships between variables
            \item \textbf{Progress}: Invariant shows we're making progress toward goal
        \end{itemize}
    \end{block}
    
    \begin{alertblock}{Remember}
        The invariant doesn't say what changes—it says what stays the same!
    \end{alertblock}
\end{frame}

\begin{frame}{Example: Complete Division Program}
    \begin{block}{From the Previous Slide}
        \begin{align*}
        \vdash \{X=R+(Y \times Q)\} \\
        \texttt{WHILE } Y \leq R \texttt{ DO} \\
        \quad (R:=R-Y; \; Q:=Q+1) \\
        \{X=R+(Y \times Q) \wedge \neg(Y \leq R)\}
        \end{align*}
    \end{block}

    \begin{block}{Setting Up the Division}
        It is easy to deduce that:
        \[\vdash \{\mathbf{T}\} \; R:=X; \; Q:=0 \; \{X=R+(Y \times Q)\}\]
    \end{block}
\end{frame}

\begin{frame}{Example: Complete Division Program}
    \begin{block}{Complete Program}
        Hence by the sequencing rule and postcondition weakening:
        \begin{align*}
        \vdash \{\mathbf{T}\} \\
        R:=X; \\
        Q:=0; \\
        \texttt{WHILE } Y \leq R \texttt{ DO} \\
        \quad (R:=R-Y; \; Q:=Q+1) \\
        \{R < Y \wedge X=R+(Y \times Q)\}
        \end{align*}
    \end{block}
\end{frame}

\begin{frame}{Summary}
    \begin{block}{What We Have Given}
        \begin{itemize}
            \item A notation for specifying what a program does
            \item A way of proving that it meets its specification
        \end{itemize}
    \end{block}
    
    \begin{block}{Next Topics}
        Now we look at ways of finding proofs and organizing them:
        \begin{itemize}
            \item Finding invariants
            \item Derived rules
            \item Backwards proofs
            \item Annotating programs prior to proof
        \end{itemize}
    \end{block}
    
    \begin{block}{Automation}
        Then we see how to automate program verification:
        \begin{itemize}
            \item The automation mechanizes some of these ideas
        \end{itemize}
    \end{block}
\end{frame}