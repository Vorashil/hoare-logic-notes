% Section 12: Formal Proofs and Sequencing
\section{Formal Proofs and the Sequencing Rule}

\begin{frame}{An Example Formal Proof}
    \begin{block}{A Little Formal Proof}
        Here is a little formal proof:
    \end{block}
    
    \begin{enumerate}
        \item $\vdash \{R=X \wedge 0=0\} \; Q:=0 \; \{R=X \wedge Q=0\}$ \hfill By the assignment axiom
        \item $\vdash R=X \Rightarrow R=X \wedge 0=0$ \hfill By pure logic
        \item $\vdash \{R=X\} \; Q:=0 \; \{R=X \wedge Q=0\}$ \hfill By precondition strengthening
        \item $\vdash R=X \wedge Q=0 \Rightarrow R=X+(Y \times Q)$ \hfill By laws of arithmetic
        \item $\vdash \{R=X\} \; Q:=0 \; \{R=X+(Y \times Q)\}$ \hfill By postcondition weakening
    \end{enumerate}
    
    \begin{block}{Note}
        The rules precondition strengthening and postcondition weakening are sometimes called the \emph{rules of consequence}
    \end{block}
\end{frame}

\begin{frame}{Analyzing the Example Proof}
    \begin{block}{What This Proof Shows}
        We proved: $\vdash \{R=X\} \; Q:=0 \; \{R=X+(Y \times Q)\}$
        \begin{itemize}
            \item Starting with $R = X$
            \item After setting $Q$ to $0$
            \item We have $R = X + (Y \times 0) = X$
        \end{itemize}
    \end{block}
    
    \begin{block}{Key Steps}
        \begin{enumerate}
            \item Started with assignment axiom for $Q := 0$
            \item Strengthened precondition from $R=X \wedge 0=0$ to just $R=X$
            \item Weakened postcondition using arithmetic ($Y \times 0 = 0$)
        \end{enumerate}
    \end{block}
    
    \begin{block}{Lesson}
        Even simple proofs often require the rules of consequence to connect axioms with desired specifications
    \end{block}
\end{frame}

\begin{frame}{The Sequencing Rule}
    \begin{block}{Syntax and Semantics}
        \begin{itemize}
            \item \textbf{Syntax}: $C_1; \cdots; C_n$
            \item \textbf{Semantics}: the commands $C_1, \cdots, C_n$ are executed in that order
            \item \textbf{Example}: $R:=X; \; X:=Y; \; Y:=R$
            \begin{itemize}
                \item The values of $X$ and $Y$ are swapped using $R$ as a temporary variable
                \item Note \emph{side effect}: value of $R$ changed to the old value of $X$
            \end{itemize}
        \end{itemize}
    \end{block}
    
    \begin{block}{The Sequencing Rule}
        \[ \frac{\vdash \{P\} \; C_1 \; \{Q\}, \quad \vdash \{Q\} \; C_2 \; \{R\}}{\vdash \{P\} \; C_1; C_2 \; \{R\}} \]
    \end{block}
\end{frame}

\begin{frame}{Understanding the Sequencing Rule}
    \begin{block}{What the Rule Says}
        \begin{itemize}
            \item If $C_1$ transforms state from $P$ to $Q$
            \item And $C_2$ transforms state from $Q$ to $R$
            \item Then $C_1; C_2$ transforms state from $P$ to $R$
        \end{itemize}
    \end{block}
    
    \begin{block}{The Middle Condition}
        \begin{itemize}
            \item $Q$ acts as a ``glue'' between the two commands
            \item It must be the postcondition of $C_1$
            \item And the precondition of $C_2$
            \item Finding the right $Q$ is often the key to sequencing proofs
        \end{itemize}
    \end{block}
    
    \begin{block}{Generalization}
        For $n$ commands: need $n-1$ intermediate conditions
        \[ \{P\} \; C_1 \; \{Q_1\} \; C_2 \; \{Q_2\} \; \cdots \; C_{n-1} \; \{Q_{n-1}\} \; C_n \; \{R\} \]
    \end{block}
\end{frame}

\begin{frame}{Example Proof: Variable Swap}
    \begin{block}{Goal}
        Prove the variable swap works correctly
    \end{block}
    
    Example: By the assignment axiom:
    \begin{align*}
        \text{(i)} \quad & \vdash \{X=x \wedge Y=y\} \; R:=X \; \{R=x \wedge Y=y\} \\
        \text{(ii)} \quad & \vdash \{R=x \wedge Y=y\} \; X:=Y \; \{R=x \wedge X=y\} \\
        \text{(iii)} \quad & \vdash \{R=x \wedge X=y\} \; Y:=R \; \{Y=x \wedge X=y\}
    \end{align*}
    
    Hence by (i), (ii) and the sequencing rule:
    \begin{align*}
        \text{(iv)} \quad & \vdash \{X=x \wedge Y=y\} \; R:=X; \; X:=Y \; \{R=x \wedge X=y\}
    \end{align*}
    
    Hence by (iv) and (iii) and the sequencing rule:
    \begin{align*}
        \text{(v)} \quad & \vdash \{X=x \wedge Y=y\} \; R:=X; \; X:=Y; \; Y:=R \; \{Y=x \wedge X=y\}
    \end{align*}
\end{frame}

\begin{frame}{Breaking Down the Swap Proof}
    \begin{block}{Step-by-Step Analysis}
        Starting with $X=x$ and $Y=y$:
        \begin{enumerate}
            \item After $R:=X$: we have $R=x$, $X=x$, $Y=y$
            \item After $X:=Y$: we have $R=x$, $X=y$, $Y=y$ 
            \item After $Y:=R$: we have $R=x$, $X=y$, $Y=x$
        \end{enumerate}
        Final result: $X$ and $Y$ are swapped!
    \end{block}
    
    \begin{block}{Key Observation}
        \begin{itemize}
            \item Each intermediate assertion captures the exact state
            \item We track all variables, including the temporary $R$
            \item The proof is compositional: we prove each step separately
        \end{itemize}
    \end{block}
    
    \begin{alertblock}{Note on Auxiliary Variables}
        The lowercase $x$ and $y$ are auxiliary variables that remember the initial values
    \end{alertblock}
\end{frame}