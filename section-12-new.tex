% Section 12: Formal Proofs and Sequencing
\section{Formal Proofs and the Sequencing Rule}

\begin{frame}{An Example Formal Proof}
    \begin{block}{A Little Formal Proof}
        Here is a little formal proof:
    \end{block}
    
    \begin{enumerate}
        \item $\vdash \{R=X \wedge 0=0\} \; Q:=0 \; \{R=X \wedge Q=0\}$ \hfill By the assignment axiom
        \item $\vdash R=X \Rightarrow R=X \wedge 0=0$ \hfill By pure logic
        \item $\vdash \{R=X\} \; Q:=0 \; \{R=X \wedge Q=0\}$ \hfill By precondition strengthening
        \item $\vdash R=X \wedge Q=0 \Rightarrow R=X+(Y \times Q)$ \hfill By laws of arithmetic
        \item $\vdash \{R=X\} \; Q:=0 \; \{R=X+(Y \times Q)\}$ \hfill By postcondition weakening
    \end{enumerate}
    
    \begin{block}{Note}
        The rules precondition strengthening and postcondition weakening are sometimes called the \emph{rules of consequence}
    \end{block}
\end{frame}

\begin{frame}{Analyzing the Example Proof}
    \begin{block}{What This Proof Shows}
        We proved: $\vdash \{R=X\} \; Q:=0 \; \{R=X+(Y \times Q)\}$
        \begin{itemize}
            \item Starting with $R = X$
            \item After setting $Q$ to $0$
            \item We have $R = X + (Y \times 0) = X$
        \end{itemize}
    \end{block}
    
    \begin{block}{Key Steps}
        \begin{enumerate}
            \item Started with assignment axiom for $Q := 0$
            \item Strengthened precondition from $R=X \wedge 0=0$ to just $R=X$
            \item Weakened postcondition using arithmetic ($Y \times 0 = 0$)
        \end{enumerate}
    \end{block}
    
    \begin{block}{Lesson}
        Even simple proofs often require the rules of consequence to connect axioms with desired specifications
    \end{block}
\end{frame}

\begin{frame}{The Sequencing Rule}
    \begin{block}{Syntax and Semantics}
        \begin{itemize}
            \item \textbf{Syntax}: $C_1; \cdots; C_n$
            \item \textbf{Semantics}: the commands $C_1, \cdots, C_n$ are executed in that order
            \item \textbf{Example}: $R:=X; \; X:=Y; \; Y:=R$
            \begin{itemize}
                \item The values of $X$ and $Y$ are swapped using $R$ as a temporary variable
                \item Note \emph{side effect}: value of $R$ changed to the old value of $X$
            \end{itemize}
        \end{itemize}
    \end{block}
    
    \begin{block}{The Sequencing Rule}
        \[ \frac{\vdash \{P\} \; C_1 \; \{Q\}, \quad \vdash \{Q\} \; C_2 \; \{R\}}{\vdash \{P\} \; C_1; C_2 \; \{R\}} \]
    \end{block}
\end{frame}

\begin{frame}{Understanding the Sequencing Rule}
    \begin{block}{What the Rule Says}
        \begin{itemize}
            \item If $C_1$ transforms state from $P$ to $Q$
            \item And $C_2$ transforms state from $Q$ to $R$
            \item Then $C_1; C_2$ transforms state from $P$ to $R$
        \end{itemize}
    \end{block}
    
    \begin{block}{The Middle Condition}
        \begin{itemize}
            \item $Q$ acts as a ``glue'' between the two commands
            \item It must be the postcondition of $C_1$
            \item And the precondition of $C_2$
            \item Finding the right $Q$ is often the key to sequencing proofs
        \end{itemize}
    \end{block}
    
    \begin{block}{Generalization}
        For $n$ commands: need $n-1$ intermediate conditions
        \[ \{P\} \; C_1 \; \{Q_1\} \; C_2 \; \{Q_2\} \; \cdots \; C_{n-1} \; \{Q_{n-1}\} \; C_n \; \{R\} \]
    \end{block}
\end{frame}

\begin{frame}{Example Proof: Variable Swap}
    \begin{block}{Goal}
        Prove the variable swap works correctly
    \end{block}
    
    Example: By the assignment axiom:
    \begin{align*}
        \text{(i)} \quad & \vdash \{X=x \wedge Y=y\} \; R:=X \; \{R=x \wedge Y=y\} \\
        \text{(ii)} \quad & \vdash \{R=x \wedge Y=y\} \; X:=Y \; \{R=x \wedge X=y\} \\
        \text{(iii)} \quad & \vdash \{R=x \wedge X=y\} \; Y:=R \; \{Y=x \wedge X=y\}
    \end{align*}
    
    Hence by (i), (ii) and the sequencing rule:
    \begin{align*}
        \text{(iv)} \quad & \vdash \{X=x \wedge Y=y\} \; R:=X; \; X:=Y \; \{R=x \wedge X=y\}
    \end{align*}
    
    Hence by (iv) and (iii) and the sequencing rule:
    \begin{align*}
        \text{(v)} \quad & \vdash \{X=x \wedge Y=y\} \; R:=X; \; X:=Y; \; Y:=R \; \{Y=x \wedge X=y\}
    \end{align*}
\end{frame}

\begin{frame}{Breaking Down the Swap Proof}
    \begin{block}{Step-by-Step Analysis}
        Starting with $X=x$ and $Y=y$:
        \begin{enumerate}
            \item After $R:=X$: we have $R=x$, $X=x$, $Y=y$
            \item After $X:=Y$: we have $R=x$, $X=y$, $Y=y$ 
            \item After $Y:=R$: we have $R=x$, $X=y$, $Y=x$
        \end{enumerate}
        Final result: $X$ and $Y$ are swapped!
    \end{block}
    
    \begin{block}{Key Observation}
        \begin{itemize}
            \item Each intermediate assertion captures the exact state
            \item We track all variables, including the temporary $R$
            \item The proof is compositional: we prove each step separately
        \end{itemize}
    \end{block}
    
    \begin{alertblock}{Note on Auxiliary Variables}
        The lowercase $x$ and $y$ are auxiliary variables that remember the initial values
    \end{alertblock}
\end{frame}

\begin{frame}{Conditionals}
    \begin{block}{Syntax and Semantics}
        \begin{itemize}
            \item \textbf{Syntax}: \texttt{IF } $S$ \texttt{ THEN } $C_1$ \texttt{ ELSE } $C_2$
            \item \textbf{Semantics}:
            \begin{itemize}
                \item If the statement $S$ is true in the current state, then $C_1$ is executed
                \item If $S$ is false, then $C_2$ is executed
            \end{itemize}
            \item \textbf{Example}: \texttt{IF X<Y THEN MAX:=Y ELSE MAX:=X}
            \begin{itemize}
                \item The value of the variable MAX is set to the maximum of the values of X and Y
            \end{itemize}
        \end{itemize}
    \end{block}
\end{frame}

\begin{frame}{The Conditional Rule}
    \begin{block}{The Conditional Rule}
        \[ \frac{\vdash \{P \wedge S\} \; C_1 \; \{Q\}, \quad \vdash \{P \wedge \neg S\} \; C_2 \; \{Q\}}{\vdash \{P\} \; \texttt{IF } S \texttt{ THEN } C_1 \texttt{ ELSE } C_2 \; \{Q\}} \]
    \end{block}
    
    \begin{block}{Understanding the Rule}
        \begin{itemize}
            \item We need to prove two things:
            \begin{itemize}
                \item When $S$ is true, $C_1$ transforms $P \wedge S$ to $Q$
                \item When $S$ is false, $C_2$ transforms $P \wedge \neg S$ to $Q$
            \end{itemize}
            \item Both branches must establish the same postcondition $Q$
            \item The precondition $P$ is strengthened by the branch condition
        \end{itemize}
    \end{block}
\end{frame}

\begin{frame}{Example: Finding the Maximum}
    \begin{block}{Goal}
        Prove: $\vdash \{\mathbf{T}\} \; \texttt{IF X$\geq$Y THEN MAX:=X ELSE MAX:=Y} \; \{MAX=\max(X,Y)\}$
    \end{block}
\end{frame}
\begin{frame}{Example: Finding the Maximum}

    \begin{block}{Step 1: Logical Facts}
        From Assignment Axiom + Precondition Strengthening:
        \begin{itemize}
            \item $\vdash (X \geq Y \Rightarrow X = \max(X,Y)) \wedge (\neg(X \geq Y) \Rightarrow Y = \max(X,Y))$
        \end{itemize}
    \end{block}
    
    \begin{block}{Step 2: Prove Each Branch}
        It follows that:
        \begin{itemize}
            \item $\vdash \{\mathbf{T} \wedge X \geq Y\} \; MAX:=X \; \{MAX=\max(X,Y)\}$
            \item $\vdash \{\mathbf{T} \wedge \neg(X \geq Y)\} \; MAX:=Y \; \{MAX=\max(X,Y)\}$
        \end{itemize}
    \end{block}
    
    \begin{block}{Step 3: Apply Conditional Rule}
        Then by the conditional rule:
        \[\vdash \{\mathbf{T}\} \; \texttt{IF X$\geq$Y THEN MAX:=X ELSE MAX:=Y} \; \{MAX=\max(X,Y)\}\]
    \end{block}
\end{frame}

\begin{frame}{Key Points about Conditionals}
    \begin{block}{Important Observations}
        \begin{itemize}
            \item Both branches must end in the same postcondition
            \item The branch condition provides extra information in each case
            \item We can use this extra information to prove different things in each branch
        \end{itemize}
    \end{block}
\end{frame}
\begin{frame}{Key Points about Conditionals}

    \begin{block}{Common Pattern}
        When proving conditional statements:
        \begin{enumerate}
            \item Identify what you know in each branch ($P \wedge S$ vs $P \wedge \neg S$)
            \item Use assignment axiom for each branch separately
            \item Apply precondition strengthening if needed
            \item Combine using the conditional rule
        \end{enumerate}
    \end{block}
    
    \begin{alertblock}{Note}
        The conditional rule requires the same postcondition $Q$ for both branches. If branches naturally lead to different postconditions, you may need to weaken them to a common $Q$.
    \end{alertblock}
\end{frame}