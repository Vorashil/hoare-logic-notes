% Section 4: Hoare Logic and Verification
\section{Hoare Logic and Verification}

\begin{frame}{Hoare Logic and Verification Conditions}
    \begin{block}{What is Hoare Logic?}
        Hoare Logic is a \textbf{deductive proof system} for Hoare triples $\hoare{P}{C}{Q}$
        \begin{itemize}
            \item Provides axioms and inference rules for proving program correctness
            \item Forms the theoretical foundation for program verification
        \end{itemize}
    \end{block}
    
    \begin{block}{Direct Verification with Hoare Logic}
        \textbf{Advantages:}
        \begin{itemize}
            \item Original proposal by Hoare
            \item Provides complete formal proofs
        \end{itemize}
        
        \textbf{Disadvantages:}
        \begin{itemize}
            \item Tedious and error-prone for humans
            \item Impractical for large programs
            \item Requires detailed manual proof construction
        \end{itemize}
    \end{block}
\end{frame}

\begin{frame}{Verification Conditions}
    \begin{block}{Definition: What is a Verification Condition?}
        A \textbf{verification condition} is a mathematical formula (without program constructs) whose truth implies the correctness of a program.
        \begin{itemize}
            \item Generated from Hoare triples by analyzing the program structure
            \item Expressed purely in terms of logic and mathematics
            \item No references to program execution or state changes
        \end{itemize}
    \end{block}
\end{frame}

\begin{frame}{Verification Conditions -2}

        \begin{block}{Modern Approach: Verification Conditions}
        Can `compile' proving $\hoare{P}{C}{Q}$ to \textbf{verification conditions}
        \begin{itemize}
            \item More natural for automated reasoning
            \item Basis for computer-assisted verification
            \item Separates program logic from mathematical reasoning
        \end{itemize}
    \end{block}
\end{frame}

\begin{frame}{Verification Conditions -3}
    \begin{block}{Key Property}
        Proof of verification conditions is \textbf{equivalent} to proof with Hoare Logic
        \begin{itemize}
            \item Hoare Logic can be used to \emph{explain} verification conditions
            \item Both approaches prove the same correctness properties
            \item Verification conditions are more amenable to automation
        \end{itemize}
    \end{block}
\end{frame}

\begin{frame}{Verification Condition Example}
    \begin{example}[Simple Verification Condition]
        To prove $\hoare{x > 0}{y := x + 1}{y > 1}$:
        
        \textbf{Step 1:} Analyze what the program does
        \begin{itemize}
            \item The assignment $y := x + 1$ sets $y$ to the value of $x + 1$
        \end{itemize}
        
        \textbf{Step 2:} Generate the verification condition
        \begin{itemize}
            \item We need: if $x > 0$ initially, then $y > 1$ after assignment
            \item Since $y$ will equal $x + 1$, we need: $x > 0 \Rightarrow (x + 1) > 1$
        \end{itemize}
        
        \textbf{Step 3:} The verification condition is:
        \[x > 0 \Rightarrow (x + 1) > 1\]
        
        This is a pure mathematical statement that can be proved using algebra, without any reference to program execution!
    \end{example}
\end{frame}