\section{Validity and Limitations of the Assignment Axiom}

\begin{frame}{Validity}
    \begin{block}{The Importance of Validity}
        Important to establish the validity of axioms and rules
    \end{block}
    
    \begin{block}{Formal Semantics and Soundness}
        Later will give a \emph{formal semantics} of our little programming language
        \begin{itemize}
            \item Then \emph{prove} axioms and rules of inference of Floyd-Hoare logic are sound
            \item This will only increase our confidence in the axioms and rules to the extent that we believe the correctness of the formal semantics!
        \end{itemize}
    \end{block}
\end{frame}

\begin{frame}{The Assignment Axiom in Real Languages}
    \begin{alertblock}{Important Limitation}
        The Assignment Axiom is not valid for `real' programming languages
    \end{alertblock}
    
    \begin{block}{Historical Note}
        In an early PhD on Hoare Logic, G. Ligler showed that the assignment axiom can fail to hold in six different ways for the language Algol 60
    \end{block}
    
    \begin{block}{Why This Matters}
        \begin{itemize}
            \item Our simple language has carefully chosen features
            \item Real languages have complications that break the axiom
            \item Understanding these limitations helps us apply Hoare Logic correctly
        \end{itemize}
    \end{block}
\end{frame}

\begin{frame}{Expressions with Side Effects}
    \begin{block}{The Hidden Assumption}
        The validity of the assignment axiom depends on expressions not having side effects
    \end{block}
    
    \begin{example}[Block Expression]
        Suppose our language were extended to contain the `block expression':
        \begin{center}
            \texttt{BEGIN Y:=1; 2 END}
        \end{center}
        \begin{itemize}
            \item This expression has value 2
            \item But its evaluation also `side effects' the variable Y by storing 1 in it
        \end{itemize}
    \end{example}
\end{frame}

\begin{frame}{Why Side Effects Break the Assignment Axiom}
    \begin{block}{The Problem}
        If the assignment axiom applied to block expressions, then it could be used to deduce:
        \[ \vdash \{Y=0\} \; X := \texttt{BEGIN Y:=1; 2 END} \; \{Y=0\} \]
    \end{block}
    
    \begin{block}{The Faulty Reasoning}
        \begin{itemize}
            \item Since $(Y=0)[E/X] = (Y=0)$ (because X does not occur in $(Y=0)$)
            \item By the assignment axiom, we'd conclude the above
            \item This is clearly false: after the assignment Y will have the value 1!
        \end{itemize}
    \end{block}
    
    \begin{alertblock}{The Lesson}
        The assignment axiom only works when expressions are \textbf{pure} (no side effects)
    \end{alertblock}
\end{frame}

\begin{frame}{Other Ways the Assignment Axiom Can Fail}
    \begin{block}{Real Language Complications}
        In real programming languages, the assignment axiom can fail due to:
        \begin{enumerate}
            \item \textbf{Side effects in expressions} (as we just saw)
            \item \textbf{Aliasing}: multiple names for the same location
            \item \textbf{Call by reference}: procedure parameters that modify variables
            \item \textbf{Global variables}: hidden dependencies between parts of code
            \item \textbf{Undefined behavior}: division by zero, array bounds violations
            \item \textbf{Concurrent modification}: other threads changing variables
        \end{enumerate}
    \end{block}
    
    \begin{block}{Defensive Programming}
        Understanding these limitations helps us:
        \begin{itemize}
            \item Design better programming languages
            \item Write more verifiable code
            \item Know when we can trust our formal proofs
        \end{itemize}
    \end{block}
\end{frame}

\begin{frame}{Example: Aliasing Breaking the Assignment Axiom}
    \begin{example}[Aliasing Problem]
        Consider if our language had arrays and we tried:
        \[ \vdash \{A[i] = 5\} \; A[j] := 0 \; \{A[i] = 5\} \]
        
        The assignment axiom would suggest this is valid because:
        \begin{itemize}
            \item $(A[i] = 5)[0/A[j]]$ might seem to be just $(A[i] = 5)$
            \item But what if $i = j$? Then $A[i]$ and $A[j]$ are the same location!
            \item After the assignment, $A[i] = 0$, not 5
        \end{itemize}
    \end{example}
    
    \begin{block}{The Solution}
        In real verification:
        \begin{itemize}
            \item Must track when different expressions might refer to same location
            \item Need more sophisticated rules for arrays and pointers
            \item This leads to \emph{separation logic} and other advanced techniques
        \end{itemize}
    \end{block}
\end{frame}

\begin{frame}{A Forwards Assignment Axiom (Floyd)}
    \begin{block}{Floyd's Original Formulation}
        This is the original semantics of assignment due to Floyd:
        \[ \vdash \{P\} \; V := E \; \{\exists v. \; V = E[v/V] \wedge P[v/V]\} \]
        where $v$ is a new variable (i.e., doesn't equal $V$ or occur in $P$ or $E$)
    \end{block}
    
    \begin{block}{What This Means}
        \begin{itemize}
            \item We start with precondition $P$
            \item After assignment, $V$ has the value that $E$ had (with old $V$ replaced by $v$)
            \item The old properties still hold (with old $V$ replaced by $v$)
            \item We use existential quantification to ``remember'' the old value
        \end{itemize}
    \end{block}
\end{frame}

\begin{frame}{Example of the Forwards Axiom}
    \begin{example}[Forwards Assignment]
        \[ \vdash \{X=1\} \; X := X+1 \; \{\exists v. \; X = X+1[v/X] \wedge X=1[v/X]\} \]
    \end{example}
    
    \begin{block}{Simplifying the Postcondition}
        \begin{align*}
            & \vdash \{X=1\} \; X := X+1 \; \{\exists v. \; X = X+1[v/X] \wedge X=1[v/X]\} \\
            & \vdash \{X=1\} \; X := X+1 \; \{\exists v. \; X = v + 1 \wedge v = 1\} \\
            & \vdash \{X=1\} \; X := X+1 \; \{\exists v. \; X = 1 + 1 \wedge v = 1\} \\
            & \vdash \{X=1\} \; X := X+1 \; \{X = 1 + 1 \wedge \exists v. \; v = 1\} \\
            & \vdash \{X=1\} \; X := X+1 \; \{X = 2 \wedge \mathbf{T}\} \\
            & \vdash \{X=1\} \; X := X+1 \; \{X = 2\}
        \end{align*}
    \end{block}
\end{frame}

\begin{frame}{Comparing Forward and Backward Axioms}
    \begin{block}{Key Observation}
        The forwards axiom is equivalent to the standard (backwards) one but harder to use
    \end{block}
    
    \begin{columns}
        \begin{column}{0.48\textwidth}
            \begin{block}{Backwards (Hoare)}
                \[ \vdash \{Q[E/V]\} \; V := E \; \{Q\} \]
                \begin{itemize}
                    \item Direct: substitute in precondition
                    \item Natural for verification
                    \item No existential quantifiers
                \end{itemize}
            \end{block}
        \end{column}
        \begin{column}{0.48\textwidth}
            \begin{block}{Forwards (Floyd)}
                \[ \vdash \{P\} \; V := E \; \{\exists v. \; V = E[v/V] \wedge P[v/V]\} \]
                \begin{itemize}
                    \item Requires existential elimination
                    \item More complex postconditions
                    \item Natural for symbolic execution
                \end{itemize}
            \end{block}
        \end{column}
    \end{columns}
\end{frame}

\begin{frame}{Why Have Two Forms?}
    \begin{block}{Different Use Cases}
        \begin{itemize}
            \item \textbf{Backwards (Hoare)}: Better for \emph{verification}
                \begin{itemize}
                    \item Start with desired postcondition
                    \item Work backwards to find required precondition
                    \item Natural for proving programs meet specifications
                \end{itemize}
            \item \textbf{Forwards (Floyd)}: Better for \emph{analysis}
                \begin{itemize}
                    \item Start with known precondition
                    \item Work forwards to compute postcondition
                    \item Natural for symbolic execution and program analysis
                \end{itemize}
        \end{itemize}
    \end{block}
    
    \begin{block}{In Practice}
        Most verification systems use Hoare's backwards form because:
        \begin{itemize}
            \item Simpler to work with (no existential quantifiers)
            \item More direct for common verification tasks
            \item Easier to automate
        \end{itemize}
    \end{block}
\end{frame}