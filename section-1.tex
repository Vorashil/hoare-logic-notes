\section{A Little Programming Language}

\begin{frame}{Syntax of the Language}
    \framesubtitle{Based on Backus-Naur Form (BNF)}

    \colorbox{yellow}{\bfseries Expressions:}
    \[ E \bnfdef N \bnfor V \bnfor E_1 + E_2 \bnfor E_1 - E_2 \bnfor E_1 \times E_2 \bnfor \dots \]

    \vspace{0.7cm}

    \colorbox{yellow}{\bfseries Boolean expressions:}
    \[ B \bnfdef \mathbf{T} \bnfor \mathbf{F} \bnfor E_1=E_2 \bnfor E_1 \le E_2 \bnfor \dots \]

    \vspace{0.7cm}

    \colorbox{yellow}{\bfseries Commands:}
    \[
        \begin{array}{lcl}
            C & \bnfdef & \assign{V}{E} \\
            & \bnfor  & \seq{C_1}{C_2} \\
            & \bnfor  & \text{IF } B \text{ THEN } C_1 \text{ ELSE } C_2 \\
            & \bnfor  & \text{WHILE } B \text{ DO } C'
        \end{array}
    \]
\end{frame}

\begin{frame}[fragile]{Example Programs - 1}
    \framesubtitle{Illustrating the language syntax}

    \begin{block}{Factorial of a number `n`}
        This program computes $n!$ and stores the result in the variable `fact`. It assumes the variable `n` holds a non-negative integer. The body of the `while` loop is a sequence of two assignment commands.
        \begin{lstlisting}[language=pseudocode, numbers=none]
fact := 1;
i := n;
while i > 0 do
    fact := fact * i;
    i := i - 1
        \end{lstlisting}
    \end{block}

    \end{frame}
\begin{frame}[fragile]{Example Programs - 2}


    \begin{block}{Maximum of two numbers `x` and `y`}
        This program uses a conditional statement to find the maximum of two numbers, `x` and `y`, and stores the result in `max`.
        \begin{lstlisting}[language=pseudocode, numbers=none]
if x <= y then
    max := y
else
    max := x
        \end{lstlisting}
    \end{block}
\end{frame}

