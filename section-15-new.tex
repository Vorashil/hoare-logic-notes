% Section 15: Derived Rules
\section{Derived Rules}

\begin{frame}{Conjunction and Disjunction}
    \begin{block}{Specification Conjunction and Disjunction}
        \textbf{Specification conjunction}
        \[ \frac{\vdash \{P_1\} \; C \; \{Q_1\}, \quad \vdash \{P_2\} \; C \; \{Q_2\}}{\vdash \{P_1 \wedge P_2\} \; C \; \{Q_1 \wedge Q_2\}} \]
        
        \textbf{Specification disjunction}
        \[ \frac{\vdash \{P_1\} \; C \; \{Q_1\}, \quad \vdash \{P_2\} \; C \; \{Q_2\}}{\vdash \{P_1 \vee P_2\} \; C \; \{Q_1 \vee Q_2\}} \]
    \end{block}
    
    \begin{block}{Use of These Rules}
        These rules are useful for splitting a proof into independent bits:
        \begin{itemize}
            \item They enable $\vdash \{P\} \; C \; \{Q_1 \wedge Q_2\}$ to be proved by proving separately that both $\vdash \{P\} \; C \; \{Q_1\}$ and also that $\vdash \{P\} \; C \; \{Q_2\}$
        \end{itemize}
    \end{block}
\end{frame}

\begin{frame}{Theoretical vs Practical Considerations}
    \begin{block}{Theoretical Status}
        Any proof with these rules could be done without using them:
        \begin{itemize}
            \item i.e., they are theoretically redundant (proof omitted)
            \item However, useful in practice
        \end{itemize}
    \end{block}
    
    \begin{block}{Why These Rules Matter in Practice}
        \begin{itemize}
            \item They make proofs more modular
            \item Allow separate verification of different properties
            \item Can simplify complex specifications
            \item Make proof structure clearer
        \end{itemize}
    \end{block}
\end{frame}

\begin{frame}{Derived Rules for Finding Proofs}
    \begin{block}{The Goal-Directed Approach}
        Suppose the goal is to prove $\{Precondition\} \; Command \; \{Postcondition\}$
        
        If there were a rule of the form:
        \[ \frac{\vdash H_1, \ldots, \vdash H_n}{\vdash \{P\} \; C \; \{Q\}} \]
        
        then we could instantiate:
        \begin{itemize}
            \item $P \mapsto Precondition$, $C \mapsto Command$, $Q \mapsto Postcondition$
            \item to get instances of $H_1, \ldots, H_n$ as subgoals
        \end{itemize}
    \end{block}
    
    \begin{block}{The Key Insight}
        \begin{itemize}
            \item Some rules are already in this form (e.g., the sequencing rule)
            \item We will derive rules of this form for all commands
            \item Then we use these derived rules for mechanizing Hoare Logic proofs
        \end{itemize}
    \end{block}
\end{frame}

\begin{frame}{Understanding Goal-Directed Proof}
    \begin{block}{What is Goal-Directed Proof?}
        Instead of building proofs from axioms up (forward), we:
        \begin{enumerate}
            \item Start with what we want to prove (the goal)
            \item Find a rule whose conclusion matches our goal
            \item The premises of that rule become our new subgoals
            \item Repeat until we reach axioms or known facts
        \end{enumerate}
    \end{block}
    
    \begin{example}[Sequencing Example]
        To prove $\{P\} \; C_1; C_2 \; \{Q\}$:
        \begin{itemize}
            \item Apply sequencing rule backwards
            \item New subgoals: find $R$ such that:
                \begin{itemize}
                    \item $\vdash \{P\} \; C_1 \; \{R\}$
                    \item $\vdash \{R\} \; C_2 \; \{Q\}$
                \end{itemize}
        \end{itemize}
    \end{example}
\end{frame}

\begin{frame}{Derived Rules}
    \begin{block}{Establishing Derived Rules for All Commands}
        We will establish derived rules for all commands:

        \begin{center}
        \begin{tabular}{c}
            $\cdots$ \\
            \hline
            $\vdash \{P\} \; V:=E \; \{Q\}$ \\
            \\
            $\cdots$ \\
            \hline
            $\vdash \{P\} \; C_1;C_2 \; \{Q\}$ \\
            \\
            $\cdots$ \\
            \hline
            $\vdash \{P\} \; \texttt{IF } S \texttt{ THEN } C_1 \texttt{ ELSE } C_2 \; \{Q\}$ \\
            \\
            $\cdots$ \\
            \hline
            $\vdash \{P\} \; \texttt{WHILE } S \texttt{ DO } C \; \{Q\}$
        \end{tabular}
        \end{center}
    \end{block}
\end{frame}
\begin{frame}{Derived Rules}

    \begin{block}{Purpose}
        These support `backwards proof' starting from a goal $\{P\} \; C \; \{Q\}$
    \end{block}
\end{frame}

\begin{frame}{The Derived Assignment Rule}
    \begin{block}{An Example Proof}
        Let's revisit our earlier proof from Section 12:
        \begin{enumerate}
            \item $\vdash \{R=X \wedge 0=0\} \; Q:=0 \; \{R=X \wedge Q=0\}$ \hfill By assignment axiom
            \item $\vdash R=X \Rightarrow R=X \wedge 0=0$ \hfill By pure logic
            \item $\vdash \{R=X\} \; Q:=0 \; \{R=X \wedge Q=0\}$ \hfill By precondition strengthening
        \end{enumerate}
    \end{block}
    
    \begin{block}{Generalizing to a Proof Schema}
        We can generalize this pattern:
        \begin{enumerate}
            \item $\vdash \{Q[E/V]\} \; V:=E \; \{Q\}$ \hfill By assignment axiom
            \item $\vdash P \Rightarrow Q[E/V]$ \hfill \colorbox{yellow}{By assumption}
            \item $\vdash \{P\} \; V:=E \; \{Q\}$ \hfill By precondition strengthening
        \end{enumerate}
    \end{block}
\end{frame}

\begin{frame}{The Derived Assignment Rule}
    \begin{block}{The Rule}
        This proof schema justifies:
        
        \begin{center}
        \fbox{
        \begin{minipage}{0.8\textwidth}
        \centering
        \textbf{Derived Assignment Rule}
        
        \vspace{0.5em}
        
        \[ \frac{\vdash P \Rightarrow Q[E/V]}{\vdash \{P\} \; V:=E \; \{Q\}} \]
        \end{minipage}
        }
        \end{center}
    \end{block}
    
    \begin{block}{Key Insight}
        \begin{itemize}
            \item $Q[E/V]$ is the \textbf{weakest liberal precondition} $wlp(V:=E, Q)$
            \item This is the weakest condition that guarantees $Q$ after $V:=E$
            \item Links back to our discussion of substitution in Section 9
        \end{itemize}
    \end{block}
\end{frame}

\begin{frame}{Understanding the Derived Assignment Rule}
    \begin{block}{Why This Rule is Powerful}
        \begin{itemize}
            \item \textbf{Goal-directed}: Start with desired postcondition $Q$
            \item \textbf{Systematic}: Compute $Q[E/V]$ mechanically
            \item \textbf{Complete}: Can derive any valid assignment triple
        \end{itemize}
    \end{block}
    
    \begin{example}[Using the Rule]
        Original proof required 3 steps:
        \begin{enumerate}
            \item $\vdash R=X \Rightarrow R=X \wedge 0=0$ \hfill By pure logic
            \item $\vdash \{R=X\} \; Q:=0 \; \{R=X \wedge Q=0\}$ \hfill By derived assignment
        \end{enumerate}
        Now only 2 steps! We saved one step by using the derived rule.
    \end{example}
\end{frame}

\begin{frame}{Weakest Liberal Precondition}
    \begin{block}{What is $Q[E/V]$?}
        Recall from Section 9: $Q[E/V]$ means ``$Q$ with $E$ substituted for $V$''
        \begin{itemize}
            \item This is exactly the precondition needed for assignment
            \item It's the \emph{weakest} condition that ensures $Q$ after $V:=E$
            \item ``Liberal'' means we don't require termination
        \end{itemize}
    \end{block}
    
    \begin{example}[Computing $wlp$]
        \begin{itemize}
            \item Goal: $\{?\} \; X:=X+1 \; \{X>0\}$
            \item Compute: $(X>0)[X+1/X] = (X+1>0) = (X>-1)$
            \item Therefore: $\vdash \{X>-1\} \; X:=X+1 \; \{X>0\}$
        \end{itemize}
    \end{example}
    
    \begin{alertblock}{Key Point}
        The derived rule tells us exactly what precondition we need!
    \end{alertblock}
\end{frame}

\begin{frame}{More Examples of Derived Assignment Rule}
    \begin{example}[Array Assignment]
        Goal: $\{?\} \; A[i]:=v \; \{A[j]=w\}$
        
        Compute $wlp$: $(A[j]=w)[A[i]:=v/A[i]]$
        
        Result depends on whether $i=j$:
        \begin{itemize}
            \item If we know $i \neq j$: precondition is $A[j]=w$
            \item If we know $i = j$: precondition is $v=w$
            \item In general: $(i=j \wedge v=w) \vee (i \neq j \wedge A[j]=w)$
        \end{itemize}
    \end{example}
    
    \begin{block}{Connection to Earlier Material}
        \begin{itemize}
            \item This shows why aliasing matters (Section 10)
            \item The derived rule handles all these cases systematically
            \item No need to worry about special cases in the proof
        \end{itemize}
    \end{block}
\end{frame}

\begin{frame}{Why Backwards Proof?}
    \begin{block}{Advantages of Working Backwards}
        \begin{itemize}
            \item \textbf{Goal-focused}: Always know what you're trying to prove
            \item \textbf{Systematic}: Each command type has a specific strategy
            \item \textbf{Mechanizable}: Can be automated more easily
            \item \textbf{Natural}: Matches how humans often think about proofs
        \end{itemize}
    \end{block}
    
    \begin{block}{The Process}
        \begin{enumerate}
            \item Look at the command structure
            \item Apply the corresponding derived rule
            \item Generate simpler subgoals
            \item Continue until reaching assignment axioms
        \end{enumerate}
    \end{block}
\end{frame}